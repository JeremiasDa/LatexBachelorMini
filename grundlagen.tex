%!TEX root = thesis.tex

\chapter{Früh- und Hochmittelalter}
\label{chapter-basics}

Dieses Kapitel beschreibt die Entstehung des römisch-deutschen Kaisertitels im Früh- bis Hochmittelalter.

\section{Ausrufung}

Die mittelalterlichen Herrscher des Reiches sahen sich – in Anknüpfung an die spätantike Kaiseridee und die Idee der \textit{Renovatio imperii}, der Wiederherstellung des Römischen Reichs unter Karl dem Großen – in direkter Nachfolge der römischen Caesaren und der karolingischen Kaiser. Sie propagierten den Gedanken der \textit{Translatio imperii}, nach dem die höchste weltliche Macht, das Imperium, von den Römern auf das fränkisch-deutsche Reich durch Gottesgnadentum übergegangen sei.

\section{Aufteilung}

Das Gebiet des frühmittelalterlichen Ostfrankenreichs wurde erstmals im 11. Jahrhundert als \textit{Regnum Teutonicum} oder \textit{Regnum Teutonicorum} (Königreich der Deutschen) bezeichnet. Bereits Otto der Große wurde 962 vom Papst zum Römischen Kaiser gekrönt, nachdem er auch den Titel eines Königs von Italien erworben hatte. Seine Nachfolger behielten diesen Anspruch bei und bestanden auf dem Recht zur Krönung zum Römischen Kaiser, das sie durch einen Krönungszug nach Italien und der Krönung durch den Papst umsetzen konnten. 1157 erscheint unter Friedrich I. erstmals der Begriff \textit{sacrum} („heilig“) für das Reich,[2] das neben dem deutschen Königreich auch das italienische und seit 1032 auch das burgundische Königreich umfasste. Die offizielle Bezeichnung als Heiliges Römisches Reich ist erstmals für 1254 belegt. Folgerichtig ließen dessen Herrscher sich selbst seit dem 11. Jahrhundert vor ihrer Kaiserkrönung \textit{Rex Romanorum} (König der Römer) nennen. Mit diesem Titel verbanden sie den Anspruch auf die Kaiserkrone und auf eine supranationale Herrschaft, die deutsche, italienische (Reichsitalien), französische und slawische Sprachgebiete umfasste. Dieser Anspruch wurde vom Papsttum seit Beginn des Investiturstreits im 11. Jahrhundert zunehmend bestritten, insbesondere durch Gregor VII. in seiner Schrift Dictatus Papae, die dem Papst die Universalherrschaft über alle geistlichen und weltlichen Herrscher zusprach.

\section{Verknüpfung an das Römische Reich}

Neben den propagandistischen gab es auch heilsgeschichtliche Gründe für die Anknüpfung des römisch-deutschen Kaisertums an das antike Römische Reich. Nach mittelalterlichem Geschichtsverständnis, das vom Kapitel 7 im Buch Daniel beeinflusst ist, hatte es in der Antike nacheinander vier Weltreiche gegeben: das der Meder, der Perser, der Griechen und der Römer. Im Römischen Reich, in dem Jesus geboren worden war und das sich seit Kaiser Konstantin zu einem Imperium Christianum gewandelt hatte, sahen viele Gelehrte seit Augustinus die endgültige Form der weltlichen Herrschaft, in der sich das Christentum bis zum Ende der Zeiten entfalten werde. Im Reich Karls des Großen und der deutschen Könige sahen sie daher nicht den Nachfolgestaat des 476 untergegangenen weströmischen Reiches, sondern dieses Reich selbst in neuer Form. Dies erklärt auch den im Hochmittelalter aufkommenden Zusatz Heilig in der offiziellen Bezeichnung des Reiches und auch des Kaisers.

Zwar bestand das Römische Reich im Osten mit dem Byzantinischen Reich (Ostrom) verfassungsrechtlich ununterbrochen fort. Da das Oströmische beziehungsweise Byzantinische Reich im Jahr 800 jedoch von einer Frau, der Kaiserin Irene regiert wurde, argumentierten die Vertreter der Translatio imperii-Theorie, der Kaiserthron sei vakant und damit vom Papst rechtmäßig auf Karl den Großen übertragen worden. Auf die karolingische Tradition berief sich 150 Jahre später wiederum Otto I., der mit der Annahme des Kaisertitels im Jahr 962 bewusst an die fränkische und die römische Reichsidee anknüpfte.

