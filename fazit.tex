%!TEX root = thesis.tex

\chapter{Zusammenfassung und Ausblick}
\label{chapter-fazit}

Wie in der Einleitung schon erwähnt befasst sich die Arbeit auf den Werdegang durch die verschiedenen Zeitalter, es wurde auf das Früh- und Hochmittelalter, Spätmittelalter und Neuzeit eingegangen. Diese Zeitalter wurden anhand ihrer Historischen Einzelheiten erklärt. Dadurch ist es dem Leser möglich, das Römische-deutsche Kaiserreich zu verstehen und nachzuvollziehen, was innerhalb dieser Zeitalter geschieht.
In diesem Artikel wurde nicht auf die einzelnen Kaiser eingegangen und deren Erfolge bzw. Niederlagen. Ein weiteres Aufgabengebiet wäre somit, die einzelnen Kaiser zu analysieren und deren Informationen die für das Thema relevant sind aufzulisten und zusammenhänge zu finden.
\cite{Schulze}
\begin{figure}[H]
\centering
\includegraphics[width = 5cm, height= 5cm]{GrossesWappen.png}
\caption{Großes Wappen des römisch-deutschen Kaisers Joseph II. (1765)}
\end{figure}