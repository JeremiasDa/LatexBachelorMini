%!TEX root = thesis.tex

\chapter{Evaluierung}
\label{chapter-evaluation}

In der Evaluierung wird das Ergebnis dieser Arbeit bewertet. Eine praktische Evaluation eines neuen Algorithmus kann zum Beispiel durch eine Implementierung geschehen. Je nach Thema der Arbeit kann sich natürlich auch die gesamte Arbeit eher im praktischen Bereich mit einer Implementierung beschäftigen. In diesem Fall gilt es am Ende der Arbeit insbesondere die Implementierung selber zu evaluieren. Wesentliche Fragen dabei können sein:
\begin{compactitem}[--]
  \item Was funktioniert jetzt besser als vor meiner Arbeit?
  \item Wie kann das praktisch eingesetzt werden?
  \item Was sagen potenzielle Anwender zu meiner Lösung?
\end{compactitem}

\section{Implementierungen}

Wenn Implementierungen umfangreich beschrieben werden, ist darauf zu achten, den richtigen Mittelweg zwischen einer zu detaillierten und zu oberflächlichen Beschreibung zu finden. Eine Beschreibung aller Details der Implementierung ist in der Regel zu detailliert, da die primäre Zielgruppe einer Abschlussarbeit sich nicht im Detail in den geschriebenen Quelltext einarbeiten will. Die Beschreibung sollte aber durchaus alle wesentlichen Konzepte der Implementierung enthalten. Gerade bei einer Abschlussarbeit am Institut für Softwaretechnik und Programmiersprachen lohnt es sich, auf die eingesetzten Techniken und Programmiersprachen einzugehen. Ich würde in einer solchen Beschreibung auch einige unterstützende Diagramme erwarten.

