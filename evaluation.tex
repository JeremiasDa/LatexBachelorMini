%!TEX root = thesis.tex

\chapter{Neuzeit}
\label{chapter-evaluation}

Seit der Annahme des Titels Erwählter Römischer Kaiser durch Maximilian I. (1508) wurde dieser von allen nachfolgenden römisch-deutschen Königen beim Antritt der Alleinherrschaft und der offiziellen Krönung verwendet, etwa durch Karl V. 1520. Auf eine Krönung durch den Papst wurde fortan verzichtet, mit Ausnahme Karls V., der sich 1530 nachträglich durch den Papst in Bologna krönen ließ.

Der letzte Kaiser des Heiligen Römischen Reiches Deutscher Nation, Franz II. führte als Titel \textit{divina favente clementia electus Romanorum Imperator, semper Augustus} („von Gottes Gnaden erwählter Römischer Kaiser, zu allen Zeiten Mehrer des Reichs“) und war nur in einem Nebentitel \textit{Germaniae Rex} („König in Germanien“; seit Maximilian I. 1508). Nachdem sich Napoleon Bonaparte selbst zum Kaiser der Franzosen proklamiert hatte, rief sich der Kaiser am 11. August 1804 als Franz I. zum Kaiser von Österreich aus, um einem Statusverlust vorzubeugen und die habsburgische Kaiserkrone weiterzuführen. Durch die Gründung des Rheinbundes unter französischem Protektorat und unter dem Druck eines französischen Ultimatums sah sich Franz II. gezwungen, am 6. August 1806 die römisch-deutsche Kaiserkrone niederzulegen. Aus Sorge, dass die Reichskrone in französische Hände gelangen könnte und die österreichischen Länder durch die lehnsrechtliche Bindung an das Reich de jure unter napoleonische Herrschaft gelangen könnten, löste er das Reich als Ganzes auf, womit er seine Kompetenzen als Reichsoberhaupt überschritt.
\cite{Schubert}

