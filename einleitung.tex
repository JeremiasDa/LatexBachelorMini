%!TEX root = thesis.tex

\chapter{Einleitung}

Die Bachelorarbeit behandelt den Römischen-deutschen Kaiser seinen Werdegang durch die verschiedenen Zeitalter. Vom früh- und Hochmittelalter über das Spätmittelalter bis zur Neuzeit, in der die Kaiserkrone niedergelegt wurde.

\section{Verwandte Arbeiten}

Weitere verwandte Arbeiten sind unter anderen das Buch von Hans K. Schulze: Grundstrukturen der Verfassung im Mittelalter. Die Literatur wurde untersucht, aber nicht verwendet in dieser Arbeit. Selbiges ist für die Zeitschrift: König und Reich. Studien zu spätmittelalterlichen deutschen Verfassungsgeschichte, diese wurden auch nicht verwendet für diese Arbeit.

\section{Aufbau der Arbeit}

Neben dieser Einleitung und der Zusammenfassung am Ende gliedert sich diese Arbeit in die folgenden drei Kapitel.
\begin{description}
  \item[\ref{chapter-basics}] beschreibt die Entstehung der Kaiseridee und der Aufbau des Kaiserreichs auf Grundlage des Byzantinischen Reichs.
  \item[\ref{chapter-konzept}] stellt die Machtverschiebung und die Wandlung der Kaiserkrönung im Spätmittelalter dar.
  \item[\ref{chapter-evaluation}] schildert den Machtverlust des Papstes und den folgenden Untergang des Römisch-deutschen Kaiserreichs.
\end{description}

