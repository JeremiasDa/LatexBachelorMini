%!TEX root = thesis.tex

\chapter{Einleitung}

Die Einleitung führt zum eigentlichen Thema dieser Arbeit hin. Dabei wird ein großer Bogen gespannt, in dem die Relevanz und der Kontext der untersuchten Thematik deutlich wird. Grundlegende Begriffe aus dem Titel und der Kurzfassung sollten aufgegriffen und definiert werden. Unterstützend können Zitate herangezogen werden, die der Arbeit einen Rahmen geben.

\section{Verwandte Arbeiten}

Eine wichtiger Abschnitt der Einleitung stellt einen Überblick über verwandte Arbeiten dar. Was wurde bereits in der Literatur untersucht und ist \emph{nicht} Thema dieser Arbeit?

\section{Aufbau der Arbeit}

Neben dieser Einleitung und der Zusammenfassung am Ende gliedert sich diese Arbeit in die folgenden drei Kapitel.
\begin{description}
  \item[\ref{chapter-basics}] beschreibt die für diese Arbeit benötigten Grundlagen. In diesem Kapitel werden \ldots, \ldots und \ldots eingeführt, da diese für die folgenden Kapitel dringend benötigt werden.
  \item[\ref{chapter-konzept}] stellt das eigentliche Konzept vor. Dabei handelt es sich um ein Konzept zur Verbesserung der Welt. Das Kapitel gliedert sich daher in einen globalen und einen lokalen Ansatz, wie die Welt zum Besseren beeinflusst werden kann.
  \item[\ref{chapter-evaluation}] beinhaltet eine Evaluation des Konzeptes aus dem vorherigen Kapitel. Anhand von Simulationen wird in diesem Kapitel untersucht, wie die Welt durch konkrete Maßnahmen deutlich verbessert werden kann.
\end{description}

