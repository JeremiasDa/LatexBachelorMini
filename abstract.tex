%!TEX root = thesis.tex

\cleardoublepage
\thispagestyle{plain}

\pdfbookmark{Kurzfassung}{kurzfassung}
\paragraph{Kurzfassung}
Als römisch-deutsche Kaiser, historische lateinische Bezeichnung \textit{Romanorum Imperator} (‚Kaiser der Römer‘), bezeichnet die neuere historische Forschung die Kaiser des Heiligen Römischen Reiches, um sie einerseits von den römischen Kaisern der Antike und andererseits von den Kaisern des Deutschen Reichs zwischen 1871 und 1918 zu unterscheiden. Ebenfalls abzugrenzen sind sie von den mittelalterlichen römischen Kaisern der Jahre 800 bis 924, deren Kaisertum seit der Teilung von Prüm auf der norditalienischen Königswürde beruhte.

\cleardoublepage
\thispagestyle{plain}

\foreignlanguage{english}{%
\pdfbookmark{Abstract}{abstract}
\paragraph{Abstract}
The Holy Roman Emperor (historically \textit{Romanorum Imperator} "Emperor of the Romans") was the ruler of the Holy Roman Empire. From an autocracy in Carolingian times the title evolved into an elected monarchy chosen by the Prince-electors. Until the Reformation the Emperor elect (\textit{imperator electus}) was required to be crowned by the Pope before assuming the imperial title.
The title was held in conjunction with the rule of the Kingdom of Germany and the Kingdom of Italy (Imperial Northern Italy). In theory, the Holy Roman Emperor was \textit{primus inter pares} (first among equals) among the other Roman Catholic monarchs; in practice, a Holy Roman Emperor was only as strong as his army and alliances made him.
Various royal houses of Europe, at different times, effectively became hereditary holders of the title, in particular in later times the Habsburgs. After the Reformation many of the subject states and most of those in Germany were Protestant while the Emperor continued to be Catholic. The Holy Roman Empire was dissolved by the last Emperor (who had additionally styled himself as the Emperor of Austria since 1804) as a result of the collapse of the polity during the Napoleonic wars.
}